\documentclass[xcolor=table]{beamer}
\usepackage{fontspec}
\usepackage{amsmath, amsfonts}
\usepackage{graphicx}
\usepackage{booktabs}
\usepackage{tikz}
\usepackage{hyperref}

\setmainfont{Times New Roman}

\usetheme{Madrid}
\usecolortheme{default}

\title{Using AI to Model Real-World Systems:\\From Weather to Economics}
\author{Seyyed Ali Mohammadiyeh\\\small Department of Pure Mathematics, University of Kashan}
\date{April 14, 2025\\\small IWDS\&AI 2025, University of Kufa, Iraq}

\begin{document}

\begin{frame}
    \titlepage
\end{frame}

\begin{frame}{Outline}
    \tableofcontents
\end{frame}

\section{Introduction}

\begin{frame}{Motivation}
    \begin{itemize}
        \item Real-world systems are dynamic and complex
        \item Traditional models often struggle with prediction
        \item AI introduces powerful tools for modeling complexity
    \end{itemize}
\end{frame}

\begin{frame}{Real-World Examples}
    \begin{itemize}
        \item Weather systems
        \item Economic fluctuations
        \item Biological ecosystems
    \end{itemize}
\end{frame}

\section{Classical Approaches}

\begin{frame}{Mathematical Modeling}
    \begin{itemize}
        \item Differential equations
        \item Dynamical systems theory
        \item Phase space analysis
    \end{itemize}
\end{frame}

\begin{frame}{Challenges with Traditional Methods}
    \begin{itemize}
        \item Sensitive dependence on initial conditions
        \item Hard to capture nonlinearities
        \item Require deep domain knowledge
    \end{itemize}
\end{frame}

\section{Artificial Intelligence in Modeling}

\begin{frame}{Why AI?}
    \begin{itemize}
        \item Learns from data
        \item Can model complex, nonlinear systems
        \item Complements domain knowledge
    \end{itemize}
\end{frame}

\begin{frame}{Types of Models}
    \begin{itemize}
        \item Feedforward Neural Networks
        \item Recurrent Neural Networks (RNNs)
        \item Long Short-Term Memory (LSTM) Networks
    \end{itemize}
\end{frame}

\begin{frame}{LSTM Networks}
    \begin{itemize}
        \item Specialized RNNs for sequences
        \item Handle long-term dependencies
        \item Widely used in time series forecasting
    \end{itemize}
\end{frame}

\section{Case Studies}

\begin{frame}{Weather Forecasting}
    \begin{itemize}
        \item Traditional: PDEs and numerical simulations
        \item AI: LSTMs trained on historical data
        \item Hybrid: Data-driven + physical models
    \end{itemize}
\end{frame}

\begin{frame}{Economic Trend Prediction}
    \begin{itemize}
        \item Stock prices, inflation, GDP trends
        \item Challenges: noise, sudden shifts
        \item AI models trained on multivariate time series
    \end{itemize}
\end{frame}

\begin{frame}{Comparison: Classical vs. AI}
    \begin{tabular}{@{}lll@{}}
        \toprule
        Feature & Classical & AI \\
        \midrule
        Interpretability & High & Low \\
        Data Dependency & Low & High \\
        Flexibility & Low & High \\
        \bottomrule
    \end{tabular}
\end{frame}

\section{Combining AI and Theory}

\begin{frame}{Hybrid Modeling Approaches}
    \begin{itemize}
        \item Neural ODEs
        \item Physics-informed Neural Networks (PINNs)
        \item Model calibration using data
    \end{itemize}
\end{frame}

\begin{frame}{Strengths of Blending Approaches}
    \begin{itemize}
        \item Leverage structure from theory
        \item Improve generalization
        \item Retain interpretability
    \end{itemize}
\end{frame}

\section{Challenges}

\begin{frame}{Limitations of AI Models}
    \begin{itemize}
        \item Require large amounts of data
        \item Susceptible to overfitting
        \item Lack of interpretability
    \end{itemize}
\end{frame}

\begin{frame}{Ethical and Practical Concerns}
    \begin{itemize}
        \item Data privacy
        \item Bias in models
        \item Reproducibility and trust
    \end{itemize}
\end{frame}

\section{Future Directions}

\begin{frame}{Trends and Opportunities}
    \begin{itemize}
        \item Explainable AI for dynamics
        \item Real-time hybrid simulations
        \item Open datasets and collaboration
    \end{itemize}
\end{frame}

\begin{frame}{Call for Collaboration}
    \begin{itemize}
        \item Bridge AI and mathematics
        \item Cross-disciplinary research
        \item Practical problem-solving
    \end{itemize}
\end{frame}

\section{Conclusion}

\begin{frame}{Summary}
    \begin{itemize}
        \item AI complements traditional models in dynamic systems
        \item LSTMs and neural networks are powerful tools
        \item Blending theory with data yields robust models
    \end{itemize}
\end{frame}

\begin{frame}{Thank You!}
    \begin{center}
        \Huge Questions?\\[1cm]
        \normalsize Seyyed Ali Mohammadiyeh\\
        Department of Pure Mathematics\\
        University of Kashan\\
        \texttt{https://github.com/BaseMax}
    \end{center}
\end{frame}

\end{document}
